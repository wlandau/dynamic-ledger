\documentclass{article}

\usepackage{amsfonts}
\usepackage{amsmath}
\usepackage{amssymb}
\usepackage{amsthm}
\usepackage{caption}
\usepackage{color}
\usepackage{enumerate}
\usepackage{fancyhdr}
\usepackage[margin=1in]{geometry}
\usepackage{hyperref}
\usepackage{graphicx}
\usepackage{latexsym}
\usepackage{listings}
\usepackage{mathrsfs}
\usepackage{natbib}
\usepackage[nottoc]{tocbibind}
\usepackage{setspace}
\usepackage{tikz}
\usepackage{tkz-graph}
\usepackage{url}

\providecommand{\all}{\ \forall \ }
\providecommand{\bs}{\backslash}
\providecommand{\e}{\varepsilon}
\providecommand{\E}{\ \exists \ }
\providecommand{\lm}[2]{\lim_{#1 \rightarrow #2}}
\providecommand{\m}[1]{\mathbb{#1}}
\providecommand{\mc}[1]{\mathcal{#1}}
\providecommand{\nv}{{}^{-1}}
\providecommand{\ov}[1]{\overline{#1}}
\providecommand{\p}{\newpage}
\providecommand{\q}{$\quad$ \newline}
\providecommand{\rt}{\rightarrow}
\providecommand{\Rt}{\Rightarrow}
\providecommand{\vc}[1]{\boldsymbol{#1}}
\providecommand{\wh}[1]{\widehat{#1}}

%\renewcommand\bibname{References}

\fancyhead{}
\fancyfoot{}
\fancyhead[R]{\thepage}
\fancyhead[C]{Landau}

\hypersetup{
    colorlinks,
    citecolor=black,
    filecolor=black,
    linkcolor=black,
    urlcolor=blue
}

\definecolor{dkgreen}{rgb}{0,0.6,0}
\definecolor{gray}{rgb}{0.5,0.5,0.5}
\definecolor{mauve}{rgb}{0.58,0,0.82}

\lstset{ 
 % basicstyle=\tiny,
  language=C,                % the language of the code
  numbers=none,
  numberfirstline=false,
  numbersep=5pt,                  % how far the line-numbers are from the code
  backgroundcolor=\color{white},      % choose the background color. You must add \usepackage{color}
  showspaces=false,               % show spaces adding particular underscores
  showstringspaces=false,         % underline spaces within strings
  showtabs=false,                 % show tabs within strings adding particular underscores
  frame=lrb,                   % adds a frame around the code
  rulecolor=\color{black},        % if not set, the frame-color may be changed on line-breaks within not-black text 
  tabsize=2,                      % sets default tabsize to 2 spaces
  captionpos=t,                   % sets the caption-position 
  breaklines=true,                % sets automatic line breaking
  breakatwhitespace=false,        % sets if automatic breaks should only happen at whitespace
  %title=\lstname,                   % show the filename of files included with \lstinputlisting;
  keywordstyle=\color{blue},          % keyword style
  commentstyle=\color{gray},       % comment style
  stringstyle=\color{dkgreen},         % string literal style
  escapeinside={\%*}{*)},            % if you want to add LaTeX within your code
  morekeywords={*, ...},               % if you want to add more keywords to the set
  xleftmargin=0.053in, % left horizontal offset of caption box
  xrightmargin=-.03in % right horizontal offset of caption box
}

\DeclareCaptionFont{white}{\color{white}}
\DeclareCaptionFormat{listing}{\parbox{\textwidth}{\colorbox{gray}{\parbox{\textwidth}{#1#2#3}}}}
\captionsetup[lstlisting]{format = listing, labelfont = white, textfont = white}
 %For caption-free listings, comment out the 3 lines above
 \lstset{frame = single}


%%% TITLE AND DATE

\title{\vspace{4cm} \hrule  \vspace{0.4cm} \huge
Dynamic Ledger: a tutorial
\vspace{0.4cm} \hrule}
\date{\today}


%%% DOCUMENT

\begin{document}
\begin{titlepage}

\maketitle

\begin{center}
\vspace{1cm}
\Large
\begin{center}
Will Landau \\ $\quad$ \\
Department of Statistics \\
Iowa State University \\ $\quad$ \\
\end{center}

\vfill
\large
Copyright \copyright ~Will Landau 2013. 
\end{center}
\end{titlepage}

\newpage 
\pagestyle{fancy}
\setcounter{page}{1}
\pagenumbering{roman}
\tableofcontents 

\newpage
\setcounter{page}{1}
\pagenumbering{arabic}
%\fancyhead[C]{\thesection}

\begin{flushleft}

\section{Introduction}

\paragraph{} Dynamic Ledger is a program for managing personal finances. Unlike most other accounting programs, it gives you control over your finances even when you have several delayed transactions. With it, you can be exactly as frugal as you need to be, and you can easily avoid spending more money than you actually have. In addition, you can use the program to clean and condense your ledgers to save space.

\section{Maintaining a ledger file}

\paragraph{} Dynamic Ledger requires you to keep your ledger in a tab-delimited spreadsheet file. This section shows you how to keep track of your transactions in this ledger file when you're using the program to balance your checkbook.

\subsection{Managing delayed transactions over time}

\paragraph{} Suppose I have two bank accounts. I began with \$800 in my first bank account, and then I make a withdrawal of \$300. Suppose it's still too early for the \$300 withdrawal to show up on my bank account's website. In addition, I have a second bank account of \$1000. I make the following tab-delimited ledger file to record this information.

\paragraph{} Let's start with the following tab-delimited ledger file, {\tt ledger.txt}.

\begin{lstlisting}[title=ledger.txt]
amount    status	credit     bank        partition    description
-300      n                  bank1
800                          bank1
1000                         bank2
\end{lstlisting}

\paragraph{} The ``n" in the status column indicates that the \$300 charge has not arrived at bank1 yet. I can use Dynamic Ledger to compute the following summary of {\tt ledger.txt}.

\begin{center}
\includegraphics[scale=.45]{fig/sum0.png}
\end{center} 

\paragraph{} The program shows me an ``available" balance of \$800.00 because that is the balance I should see when I log on to my bank account's website to check. However, my true balance is \$500 because of the delayed \$300 withdrawal. 

\paragraph{} Suppose that next time I log on to my bank account's website, the \$300 withdrawal is shown as ``pending". To make {\tt ledger.txt} agree with what I see online, I change the ``n" status to ``p":

\begin{lstlisting}[title=ledger.txt]
amount    status	credit     bank        partition    description
-300      p                  bank1
800                          bank1
1000                         bank2
\end{lstlisting}

\paragraph{} The summary from Dynamic Ledger is now

\begin{center}
\includegraphics[scale=.45]{fig/sum1.png}
\end{center}  

\paragraph{} When the withdrawal finally clears online, I can delete the ``p":

\begin{lstlisting}[title=ledger.txt]
amount    status	credit     bank        partition    description
-300                         bank1
800                          bank1
1000                         bank2
\end{lstlisting}

\paragraph{} The summary from Dynamic Ledger is now

\begin{center}
\includegraphics[scale=.45]{fig/sum2.png}
\end{center}  





\subsection{Managing credit accounts}

\paragraph{} Under the program's conceptual model, transactions flow through credit accounts into bank accounts. To see how this works, consider the following extended example. Suppose I start with a checking account with \$800. I spend \$5 on paper, then \$15.36 on food, and then \$30.14 on gas. I pay all those things with a credit card, but it's too early for any of those transactions to actually show up my credit card's website. I make the following tab-delimited ledger file.

\begin{lstlisting}[title=ledger.txt]
amount    status	credit     bank        partition    description
-30.14    cn      card       checking                 gas
-15.36    cn      card       checking                 food
-5        cn      card       checking                 paper
800                          checking
\end{lstlisting}

\paragraph{} Note the ``cn" transaction status code for all three charges. That means I made these transactions with a credit card, but it's too early for the charges to actually show up on the credit account's website. The summary of the ledger is \q

\begin{center}
\includegraphics[scale=.45]{fig/sum3.png}
\end{center} 

\paragraph{} Notice that my credit card has an ``available" balance of \$0.00, but my true balance is -\$50.50. That means that when I go on online, I should see an account balance of  \$0.00 (if I recorded my transactions correctly). However, because I have delayed transactions, I owe the credit company \$50.50 in reality. Similarly, my bank account should show an ``available" balance of \$800.00 online, but in reality, I only have \$749.50 left to spend. 

\paragraph{} Over time, transactions will begin clear on the credit card company's website. Suppose that after a few days I see that the food and paper transactions have cleared, but the gas transaction is still ``pending". I change my transaction status codes to reflect the changes.

\begin{lstlisting}[title=ledger.txt]
amount    status	credit     bank        partition    description
-30.14    cp      card       checking                 gas
-15.36    c       card       checking                 food
-5        c       card       checking                 paper
800                          checking
\end{lstlisting}

\paragraph{} ``cp" means pending on the credit card, while ``c" means charged to the credit card but unpaid. The new summary from the program looks like \q

\begin{center}
\includegraphics[scale=.45]{fig/sum4.png}
\end{center} 

\paragraph{} Since some charges have cleared, I can now make a credit card payment. I now pay my ``available" debt of \$20.36. (I cannot pay for pending charges). I now think of the food and paper transactions as a single charge of \$20.36 en route to my checking account. When I make the payment and it clears on the credit company's website, I update the ledger file.

\begin{lstlisting}[title=ledger.txt]
amount    status	credit     bank        partition    description
-30.14    cp      card       checking                 gas
-15.36    n       card       checking                 food (cred pmnt $20.36)
-5        n       card       checking                 paper (cred pmnt $20.36)
800                          checking
\end{lstlisting}

\paragraph{} The ``n" statuses means that the credit payment has not shown up on my bank account's website yet. The summary from the program is now \q

\begin{center}
\includegraphics[scale=.45]{fig/sum5.png}
\end{center} \q

\paragraph{} I wait a day or two, and then I log on again and see that my credit card payment shows up as ``pending". Now, I change the n's to p's.

\begin{lstlisting}[title=ledger.txt]
amount    status	credit     bank        partition    description
-30.14    cp      card       checking                 gas
-15.36    p       card       checking                 food (cred pmnt $20.36)
-5        p       card       checking                 paper (cred pmnt $20.36)
800                          checking
\end{lstlisting}

\paragraph{} My summary shows

\begin{center}
\includegraphics[scale=.45]{fig/sum6.png}
\end{center} \q

\paragraph{} Finally, when the credit card payment clears, I can delete the p's to show that the food and paper charges have completely cleared all my accounts.

\begin{lstlisting}[title=ledger.txt]
amount    status	credit     bank        partition    description
-30.14    cp      card       checking                 gas
-15.36            card       checking                 food (cred pmnt $20.36)
-5                card       checking                 paper (cred pmnt $20.36)
800                          checking
\end{lstlisting}

\paragraph{} My updated summary is now

\begin{center}
\includegraphics[scale=.45]{fig/sum7.png}
\end{center} \q



\subsection{Partitioning bank accounts}

\paragraph{} The program lets the user divide bank accounts into partitions. For example, if I make special partitions in my bank account for food and gas, I might write

\begin{lstlisting}[title=ledger.txt]
amount    status	credit     bank        partition    description
-30.14    cp      card       checking    gas          gas
-15.36            card       checking    food         food (cred pmnt $20.36)
-5                card       checking                 paper (cred pmnt $20.36)
400                          checking    
300                          checking    food
100                          checking    gas
\end{lstlisting}

\paragraph{} And my summary would look like 

\begin{center}
\includegraphics[scale=.45]{fig/sum8.png}
\end{center} \q

\paragraph{} Notice that only the true final balances for the partitions are shown. In other words, partition balances are calculated as if all transactions have completely cleared. This feature encourages you to avoid spending more money than you actually have.



\section{Installing the program}

\section{Using the program}





\end{flushleft}
%\newpage 
%\bibliographystyle{plainnat} 
%\bibliography{}
\end{document}